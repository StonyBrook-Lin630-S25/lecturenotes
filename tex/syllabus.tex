\setcounter{chapter}{-1}
\chapter{Syllabus}
\label{cha:syllabus}
\setcounter{page}{1}
\pagestyle{fancy}

\fcolorbox{gray!25}{gray!25}{
    \centering
    \begin{tabular}{l@{\qquad}l}
        \textbf{Course:} Parsing and Processing &
        \textbf{Name:} Thomas Graf\\
        \textbf{Course\#:} LIN630 &
        \textbf{Email:} lin630@thomasgraf.net\\
        \textbf{Time:} MW 9:30--10:50 &
        \textbf{Office hours:} M 2:00-3:30, W 2:00-2:30, F 9:30-10:30\\
        \textbf{Location:} CompLab SBS N250&
        \textbf{Office:} SBS N249\\
    \end{tabular}
}

\bigskip
\noindent
\fcolorbox{gray!25}{gray!25}{
    \centering
    \begin{tabular}{r@{\qquad}l}
        \textbf{Lecture notes}
        & \url{http://lin630.thomasgraf.net/lecturenotes}\\
        \textbf{Zoom:}
        & \url{http://lin630.thomasgraf.net/zoom}\\
        \textbf{Recordings \& other resources:}
        & \url{http://lin630.thomasgraf.net/resources}\\
    \end{tabular}
}

\section{Bulletin description (the legalese)}

A survey of parsing theory for natural language processing and its applications in psycholinguistic modeling. 
The course covers a wide variety of parsing algorithms for context-free and mildly context-sensitive grammar formalisms.
The performance of these algorithms is carefully analyzed and set in relation to empirical phenomena of human sentence processing.

\section{Overview}
\begin{itemize}
    \item \textbf{Big questions}
        \begin{itemize}
            \item What is the relation between competence and performance, grammar and parser?
            \item Are sentence processing effects conditioned by the grammar?
            \item Can syntactic constraints be reduced to processing effects?
            \item What even is a parser? Do humans actually parse?
        \end{itemize}

    \item \textbf{Learning objectives}\\
        At the end of this course you will 
        \begin{itemize}
            \item be familiar with a variety of common parsing models (top-down, bottom-up, left-corner, Earley, CYK)
            \item know the most common syntactic processing effects (in particular those related to memory usage)
            \item be able evaluate claims in the psycholinguistic literature from a computational perspective
        \end{itemize}

    \item \textbf{Prerequisites}\\
    None beyond basic syntax skills --- you should be able to draw a reasonable tree for a sentence like \emph{The fact that the employee who the manager hired stole office supplies did not go unnoticed by the janitor}.

    Knowledge of mathematics (as covered in LIN 539 \emph{Mathematical Methods}) and theoretical computational linguistics (as covered in LIN 637 \emph{Computational Linguistics 2}) is helpful at various points, but not necessary.
\end{itemize}

\section{What you have to do}

The specific work you will be doing for this course depends on \textsc{i}) the number of credits you are taking the course for, and \textsc{ii}) whether you would rather do a research project or a coding project.

\begin{itemize}
    \item Research projects depend on your specific interests.
          The default is to apply the MG processing model to a specific syntactic construction.
          Ideally, this will result in a paper or abstract (your choice).
    \item Coding projects will contribute to the \texttt{mgproc} Python package.
          Ideally, this will result in a published paper in JOSS, the \emph{Journal of Open Source Software}.
\end{itemize}

\begin{itemize}
    \item \textbf{0 credits}: have fun!
    \item \textbf{1 credits}: the above + regular attendance
    \item \textbf{2 credits}: the above + minor contribution to \texttt{mgproc} (e.g.~unit test, documentation)
    \item \textbf{3 credits}: the above +
        \begin{itemize}
            \item \emph{research track}: abstract/paper for specific conference and/or journal
            \item \emph{programming track}: major contribution to \texttt{mgproc}
        \end{itemize}
\end{itemize}


\section{Outline}

In contrast to previous years, we will try to get to MG parsing as quickly as possible.
This will be a steep learning curve, but hey, students seem to struggle with MG parsing no matter which week I cover it in.

The second half of the semester will be more eclectic and also feature guest lectures sprinkled throughout.

\begin{center}
    \begin{tabular}{r@{\hspace{2em}}l@{\hspace{2em}}l@{\hspace{2em}}l}
        \toprule
        \textbf{Wk} & \textbf{Chap} & \textbf{Topic}\\
        \midrule
        1 & \ref{cha:syllabus}--\ref{cha:ParserOverview}  & Organization, big picture, modular view of parsing\\
        2 & \ref{cha:TopDown}--\ref{cha:TopDownEval}      & Top-down parsing, predictions for sentence processing\\
        3 & readings from the literature                  & MGs Intro, parsing prelims\\
        4 & \ref{cha:MGTopDown},\ref{cha:StablerParser}   & MG top-down parsing\\
        5 & \ref{cha:BottomUp}                            & Bottom-up parsing\\
        6 & \ref{cha:LeftCorner}                          & (Generalized) Left-corner parsing\\
        7 & same                                          & same\\
        \midrule
        8 & Spring break                                  & \\
        \midrule
        9 & readings from the literature                  & MG Left-corner parsing\\
        10 & \ref{cha:ChartParsing}                       & Chart parsing overview, CKY\\
        11 & \ref{cha:Earley}                             & Earley parsing\\
        12--15 & buffer                                   & \\
        \bottomrule
    \end{tabular}
\end{center}

\section{Policies}

\subsection{Contacting me}
\begin{itemize}
    \item Emails should be sent to \href{mailto://lin630@thomasgraf.net}{lin630@thomasgraf.net} to make sure they go to my high priority inbox.
    \item Reply time $<24$h in simple cases, possibly more if meddling with bureaucracy is involved.
    \item If you want to come to my office hours and anticipate a longer meeting, please email me so that we can set aside enough time and avoid collisions with other students.
\end{itemize}

\input{./tex/blabla.tex}
